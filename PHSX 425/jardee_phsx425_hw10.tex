\documentclass[12pt]{article}

\usepackage[english]{babel}

% Math/Greek packages
\usepackage{amssymb,amsmath,amsthm, mathtools} 
\usepackage{algorithm, algorithmic}
\usepackage{upgreek, siunitx}

% Graphics/Presentation packages
\usepackage{geometry, graphicx}
\usepackage{tabulary, enumitem, array}
\usepackage{xparse,mleftright,tikz}
\usepackage{physics}

% Misc packages
\usepackage{fancyhdr}


\usepackage[export]{adjustbox}

\usepackage{esint}

\sisetup{locale=US,group-separator = {,}}
\usepackage[colorlinks=true, allcolors=blue]{hyperref}


% Box function - update this as more sophisticated solutions are found
\newcommand\mybox[2][]{\tikz[overlay]\node[fill=blue!20,inner sep=2pt, anchor=text, rectangle, rounded corners=1mm,#1] {#2};\phantom{#2}}
\renewcommand{\arraystretch}{1.2}

% General macro declarations


\makeatletter
\let\oldabs\abs
\def\abs{\@ifstar{\oldabs}{\oldabs*}}
%
\let\oldnorm\norm
\def\norm{\@ifstar{\oldnorm}{\oldnorm*}}
\makeatother

\begin{document}

\title{PHSX 425: HW10}
\author{William Jardee}
\maketitle

\section*{Question 1}
\emph{In class, and also in Griffiths, the reflection and transmission coefficients at normal incidence were derived assuming that $\mu = \mu_0$. Derive the reflection and transmission coefficients without making this assumption, and then show that energy is conserved $($i.e., $R + T = 1)$.}



%--------------------------------------------------------------
\newpage

\section*{Question 2}
\emph{Check your results to the previous problem against the solutions derived in Griffiths for oblique incidence, with the polarization in the plan of incidence. Show that $R + T = 1$ in this more general case.}



%--------------------------------------------------------------
\newpage

\section*{Question 3}
\emph{We have assumed that the wave frequency, $\omega$, is constant across an interface. Consider a wave function of the form $\psi_I = f_I(\vb{r})e^{-i\omega_I t}$, incident on a boundary from material 1 to material 2. The incident, relfected, and transmitted waves in these two substances must satify a generalized linear wave equation,}
\[[a_j\laplacian + b_j \partial_t + c_j\partial^2]\psi = 0\]
\emph{Where $j=1,2$ denotes materials 1 and 2. Show that $\omega_I = \omega_R =\omega_T$. Were any addition assumptions necessary to show this?\footnote{Hint: you might find it helpful to solve problem 9.16 first. Consider reflected and transmitted waves of the same form, and assume some general sort of boundary condition holds.} Would your conclusion change for a nonlinear wave equation? Why or why not?\footnote{Hint: If the wave equation were nonlinear in $\psi$, would solutions of the assumed form be likely to work?}}



\end{document}