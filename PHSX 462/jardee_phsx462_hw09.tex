\documentclass[11pt]{article}

\usepackage[english]{babel}
\usepackage[margin=1in]{geometry}

% Math/Greek packages
\usepackage{amssymb,amsmath,amsthm, mathtools} 
\usepackage{algorithm, algorithmic}
\usepackage{upgreek, siunitx}
\usepackage{setspace}

% Graphics/Presentation packages
\usepackage{multirow}
\usepackage{graphicx}
\usepackage{cancel}
\usepackage{tabulary, enumitem, array}
\usepackage{xparse,mleftright,tikz}
\usepackage{physics}

% Misc packages
\usepackage{fancyhdr}


\usepackage[export]{adjustbox}

\usepackage{esint}

\sisetup{locale=US,group-separator = {,}}
\usepackage[colorlinks=true, allcolors=blue]{hyperref}


\begin{document}

\title{PHSX 462: HW09}
\author{William Jardee}
\maketitle


\section*{Griffiths 11.2}
Let's start here by simply copying the wavefunctions that we will need:
\begin{align*}
\psi_{100} & = \frac{1}{\sqrt{\pi a^3}}e^{-r/a} \, ,\\
\psi_{200} & = \frac{1}{\sqrt{8\pi a^3}}\left(1 - \frac{r}{2a}\right)e^{-r/2a} \, ,\\
\psi_{210} & = \frac{1}{\sqrt{32\pi a^3}}\frac{r}{a} e^{-r/2a}\cos(\theta) \, ,\\
\psi_{21\pm 1} & = \mp \frac{1}{\sqrt{64 \pi a^3}}\frac{r}{a} e^{-r/2a}\sin(\theta)e^{\pm i\phi} \, .
\end{align*}
Since $r$ has the same degree as $z$ ($r^2 = x^2 + y^2 + z^2$), we can just analyze each as even or odd and eliminate each on that. For each of the $H^\prime_{ii}$, they are even in $z$; thus, each is zero. The same holds for each of the combinations of $\ket{200}, \ket{211}, \ket{21-1}$ with $\ket{100}$. However, the combination of $\ket{210}$ with $\ket{100}$ is not, so let us look at that:
\begin{align*}
H^\prime_{100, 210} & = eE\frac{1}{\sqrt{\pi a^3}}\frac{1}{\sqrt{32 \pi a^3}}\frac{1}{a}\int re^{-r/a} e^{-r/2a}\cos(\theta) z r^2 \sin(\theta) \dd{r}\dd{\theta}\dd{\phi}\\
& = eE \frac{1}{a^4\sqrt{8}\pi}\int^\infty_0 r^4 e^{-3r/2a} \dd{r} \int_0^\pi \cos[2](\theta) \dd{\theta} \int^{2\pi}_0 \dd{\phi}\\
& = \frac{eE}{2\sqrt{2}a^4}4!\left(\frac{2a}{3}\right)^5\frac{2}{3}\\
& = \left(\frac{2^{17/2}}{3^5}\right)eEa \, .
\end{align*}
So, every matrix element is zero except 
\[\boxed{H^\prime_{100, 210} = \left(\frac{2^{17/2}}{3^5}\right)eEa} \, .\]

%--------------------------------------------------------------------------------
\newpage

\section*{Griffiths 11.8}
\begin{enumerate}[label=\alph*)]
\item According to first order perturbation theory, we are looking at
\begin{align*}
c_a^{(1)}(t) & = 1\\
c_b^{(1)}(t) & = -\frac{i}{\hbar} \int_{t_0}^t H^\prime_{ba}(t^\prime)e^{i\omega_0 t^\prime}\dd{t^\prime}\\
& = - \frac{i}{\hbar}\int_{-\infty}^t\frac{\alpha}{\sqrt{\pi}\tau} e^{-(t^\prime/\tau)^2}e^{i\omega t^\prime} \dd{t^\prime}\\
& = -\frac{i}{\hbar}\frac{\alpha}{\sqrt{\pi}\tau}\int_{-\infty}^t e^{-\frac{1}{\tau^2}\left(t^\prime - \frac{i\omega\tau^2}{2}\right)^2 - \frac{\omega^2 \tau^2}{4}}\, 
\end{align*}
taking into consideration that we will be taking $t \rightarrow \infty$
\begin{align*}
c_b^{(1)}(t) & = - \frac{i\alpha}{\hbar \sqrt{\pi}}e^{-(\omega_0 \tau/2)^2}\sqrt{\pi}\\
& = - \frac{i\alpha}{\hbar}e^{-(\omega_0 \tau/2)^2} \, .
\end{align*}
The probability of the transition will then be 
\[\boxed{P_{a\rightarrow b} = \left(\frac{\alpha}{\hbar}\right)^2 e^{-(\omega_0 \tau)^2/2}} \, .\]

\item Applying a slight modification on the previous part:
\[c_b^{(1)} = -\frac{i}{\hbar}\int_{-\infty}^\infty \alpha \delta(t)e^{i\omega t}\dd{t} = -\frac{i\alpha}{\hbar} \, .\]
This gives a probability of $(\alpha/\hbar)^2$, which is the first order term of the solution of 11.4 ($\sin[2](\alpha/\hbar)$).

\item The probability gets killed off, as $e^{-(\omega \tau)^2/2}\rightarrow 0$. Consequently, we see no transition of states, hinting at an adiabatic-type system.

\end{enumerate}

%--------------------------------------------------------------------------------
\newpage

\section*{Griffiths 11.9}
\textit{This is post homework Will typing this up; this question was a pain in the butt while doing it, but actually super fun on the other side. Since I am straight exhausted, I am going to be skipping some steps to save my sanity...}
\begin{enumerate}[label=\alph*)]
\item Plugging in what we are given
\begin{align*}
\dot{c}_a & = -\frac{i}{\hbar}H^\prime_{ab}e^{-i\omega_0 t}c_b  = -\frac{i}{2\hbar}V_{ab}e^{i(\omega - \omega_0)t}c_b\\
\dot{c}_b & = -\frac{i}{\hbar}H^\prime_{ba}e^{i\omega_0 t}c_a  = -\frac{i}{2\hbar}V_{ab}e^{i(\omega_0 - \omega)t}c_a \, .
\end{align*}
We must go down before we can go up, so let us differentiate $\dot{c_a}$:
\begin{align*}
\ddot{c}_a &= -\frac{i}{2\hbar}V_{ab}\left[i(\omega-\omega_0)e^{i(\omega-\omega_0)t}c_b + e^{i(\omega-\omega_0)t}\dot{c}_b\right]\\
& = - \frac{i}{2\hbar}V_{ab}\left[i(\omega-\omega_0)\frac{i2\hbar}{V_{ab}}\dot{c}_a + e^{i(\omega-\omega_0)t}\left(-\frac{i}{2\hbar}\right)V_{ba}e^{i(\omega_0-\omega)t}c_a\right]\\
& = i(\omega-\omega_0)\dot{c}_a - \frac{\abs{V_{ab}}^2}{(2\hbar)^2}c_a \, . 
\end{align*}
Rewriting this gives
\[\ddot{c}_a + i(\omega_0-\omega)\dot{c}_a +\frac{\abs{V_{ab}}^2}{(2\hbar)^2}c_a = 0 \, . \]
This is a differential equation, that if we propose an exponential solution in the form of $e^{\lambda t}$, gives
\[\lambda^2 + i(\omega_0-\omega)\lambda +\frac{\abs{V_{ab}}^2}{(2\hbar)^2} = 0\]
and, via the quadratic formula
\[\lambda = -\frac{1}{2} \left[i(\omega-\omega_0)\pm \sqrt{-(\omega_0-\omega)^2 - \frac{4\abs{V_{ab}}^2}{(2\hbar)^2}}\right] = -i\left[\frac{(\omega - \omega_0)}{2} \pm \omega_r\right] \, .\]
the general equation for $c_a$ is then
\[c_a(t) = Ae^{-i((\omega - \omega_0)/2 + \omega_r)t} + Be^{-i((\omega - \omega_0)/2 - \omega_r)t} \, .\]
Using the initial conditions gives $A+B = 1$.

By a very similar derivation, the general solution for $c_b$ is 
\[c_a(t) = Ce^{-i((\omega - \omega_0)/2 + \omega_r)t} + De^{-i((\omega - \omega_0)/2 - \omega_r)t} \, .\]
Using the initial conditions here gives $C+D = 0$.

To solve for the coefficients in the $c_b$ equation, let us take the derivative: 
\begin{align*}
\dot{c}_b & = -i\left[\frac{(\omega - \omega_0)}{2} + \omega_r\right]Ce^{-i\left[\frac{(\omega - \omega_0)}{2} + \omega_r\right]} \\
& \hspace{5em} -i\left[\frac{(\omega - \omega_0)}{2} - \omega_r\right]De^{-i\left[\frac{(\omega - \omega_0)}{2} - \omega_r\right]}\\
-\frac{i}{2\hbar}V_{ba}e^{i(\omega_0-\omega)t}c_a & = \qquad " \quad "\\
c_a & = \left[\frac{2\hbar}{V_{ba}}\right]\left[\left(\frac{(\omega-omega_0)}{2} + \omega_r\right)Ce^{-i\omega_r t} \right. \\
& \hspace{5em } \left. + \left(\frac{(\omega-omega_0)}{2} - \omega_r\right)De^{i\omega_r t}\right]\\
1 & = \left[\frac{2\hbar}{V_{ba}}\right]\left[\left(\frac{(\omega-omega_0)}{2} + \omega_r\right)C + \left(\frac{(\omega-omega_0)}{2} - \omega_r\right)D\right]\\
1 & = \left[\frac{2\hbar}{V_{ba}}\right][C-D]\omega_r
\end{align*}
\[C - D = \frac{V_{ba}}{2\hbar \omega_r} \, .\]
Combining this with an initial condition gives
\[2C = \frac{V_{ba}}{2\hbar \omega_r} \rightarrow C = \frac{V_{ba}}{4\hbar \omega_r} \, \text{ and } \, D = -\frac{V_{ba}}{4\hbar \omega_r} \, .\]
At this point I am realizing my flaw in keeping everything in complex exponentials. On my scratch work I worked out what this is, specifically when I plugged in $C$ and $D$; 
\[c_b(t)  = -\frac{i}{2\hbar \omega_r}V_{ba} \left[e^{i((\omega-\omega_0)/2)t}\right]\sin(\omega_r t) \, .\]

A very similar process will need to be done with $c_a$. I will skip typing it all up, but we get
\begin{align*}
c_b & = \frac{V_{ab}}{2\hbar}\left[\left(\frac{\omega_0 - \omega}{2} + \omega_r\right)Ae^{-i\omega_r t} + \left(\frac{\omega_0 - \omega}{2} - \omega_r\right)Be^{i\omega_r t}\right]\\
0 & = \frac{V_{ab}}{2\hbar}\left[\left(\frac{\omega_0 - \omega}{2} + \omega_r\right)A + \left(\frac{\omega_0 - \omega}{2} - \omega_r\right)B\right]\\
0 & = \frac{\omega_0-\omega}{2}+ \omega_r[A-B] \, .
\end{align*}
Combining this equation with the initial values from before we get
\[2A = 1 + \frac{\omega-\omega_0}{2\omega_r} \rightarrow A = \frac{1}{2} + \frac{\omega-\omega}{4\omega_r} \, \text{ and } \, B = \frac{1}{2} - \frac{\omega-\omega}{4\omega_r} \, .\]
Plugging these into $c_a$, and doing some simplifying,
\[c_a(t) = e^{-i((\omega_0 - \omega)/2)t}\left[\cos(\omega_r t) + \frac{i(\omega_0 - \omega)}{2\omega_r}\sin(\omega_r t)\right] \, .\]

Finally, the two constants are 
\[\boxed{\mqty{c_a(t) = e^{-i((\omega_0 - \omega)/2)t}\left[\cos(\omega_r t) + \frac{i(\omega_0 - \omega)}{2\omega_r}\sin(\omega_r t)\right] \\ c_b(t)  = -\frac{i}{2\hbar \omega_r}V_{ba} \left[e^{i((\omega-\omega_0)/2)t}\right]\sin(\omega_r t)}}\]

\item To show that the probability of state $b$ is never larger than 1, let us take the square of the component:
\begin{align*}
\abs{c_b(t)}^2 & = \frac{1}{4\hbar^2\omega_r^2}\abs{V_{ab}}^2 \sin[2](\omega_r t) \rightarrow \left(\frac{\abs{V_{ab}}}{2\hbar \omega_r}\right)^2 \, .
\end{align*}
We need to compare the top: $(\abs{V_{ab}}/2\hbar)^2$ to the bottom: $\displaystyle{\frac{1}{4}\left((\omega-\omega_0)^2 + (\abs{V_{ab}}/2\hbar)^2\right)}$. Since the bottom will always be larger than the top, the probability never goes above 1. 

To show that the total probability is one, let us take the sum of the square of the components:
\[\cos[2](\omega_r t) + \left(\frac{\omega_0 - \omega}{2\omega_r}\right)^2 \sin[2](\omega_r t) + \frac{1}{4\hbar^2 \omega^2_r}\abs{V_{ab}}^2\sin[2](\omega_r t)\]
\[\cos[2](\omega_r t) + \left(\frac{1}{2\omega_r}\right)^2 \left((\omega_0-\omega)^2 + (\abs{V_{ab}}/2\hbar)^2\right)\sin[2](\omega_r t)\]
\[\cos[2](\omega_r t) + \frac{\omega_r^2}{\omega_r^2} \sin[2](\omega_r t)\]
\[\cos[2](\omega_r t) + \sin[2](\omega_r t) = t \qquad \checkmark\]

\item 
\begin{align*}
\abs{c_b(t)}^2 & = \left(\frac{\abs{V_{ab}}^2}{2\hbar}\right)^2 \frac{4}{(\omega-\omega_0)^2 + (\abs{V_{ab}}/2\hbar)^2}\sin[2](\frac{1}{2}\sqrt{(\omega-\omega)^2 + (\abs{V_{ab}}/\hbar)^2})\\
& \qquad \text{ if } (\omega - \omega_0)^2 \gg (\abs{V_{ab}}/\hbar)^2\\
& = \left(\frac{\abs{V_{ab}}^2}{2\hbar}\right)^2 \frac{4}{(\omega-\omega_0)^2}\sin[2](\frac{1}{2}(\omega-\omega))\\
& = \frac{\abs{V_{ab}}^2}{\hbar^2 (\omega-\omega_0)^2}\sin[2]((\omega-\omega_0)/2) \, .
\end{align*}
This is the solution that we got in the perturbation theory. As already stated, the definition of small here is that $(\omega - \omega_0)^2 \gg (\abs{V_{ab}}/\hbar)^2$. 

\item The state is cycles on $\omega_r t$. So, the question when what have made one half rotation (since sign doesn't matter here), so $\omega_r t = \pi \rightarrow \boxed{t = \pi/\omega_r}$ . 

\end{enumerate}

%--------------------------------------------------------------------------------
\newpage

\section*{Griffiths 11.11}
\begin{enumerate}[label=\alph*)]
\item We want to find the solution to 
\[\left(\frac{1}{c^2}\pdv[2]{t} - \nabla^2\right)f(x,y,z,t) = 0 \, .\]
Let us start by assuming that the equation is separable, 
\[f(x,y,z,t) = f_x(x)f_y(y)f_z(z)f_t(t) \, ,\]
and plug it back into the differential equation,
\[\frac{1}{c^2}f^{\prime\prime}_t f_xf_yf_z - f^{\prime\prime}_xf_yf_zf_t - f^{\prime\prime}_yf_xf_zf_t - f^{\prime\prime}_zf_xf_yf_t = 0 \rightarrow \frac{1}{c^2}\frac{f^{\prime \prime}_t}{f_t} - \frac{f^{\prime \prime}_x}{f_x} - \frac{f^{\prime \prime}_y}{f_y} - \frac{f^{\prime \prime}_z}{f_z} = 0\, ,\]
\[\frac{1}{c^2}\frac{f^{\prime \prime}_t}{f_t} = \frac{f^{\prime \prime}_x}{f_x} + \frac{f^{\prime \prime}_y}{f_y} + \frac{f^{\prime \prime}_z}{f_z} \, .\]
The right and left sides must always be equal (isolating the temporal and spatial functions), so if we let this be some $-\alpha$, 
\[f^{\prime \prime}_t = -\alpha c^2 f_t \rightarrow C_1 \cos(\omega t) + C_2 \sin(\omega t) \text{ where } \omega = \sqrt{\alpha}c \, .\]
If we let the function start at the apex, then we can just choose the cosine part. 

All together, the three spatial functions must be also be provide this $-\alpha$, but individually must be a constant so they can vary independently. putting together this idea gives that each function can be written as
\begin{align*}
f_x^{\prime \prime} & = -k_x^2 f_x & f_y^{\prime \prime} & = -k_y^2 f_y & f_z^{\prime \prime} & = -k_z^2 f_z \, ,
\end{align*}
which gives
\begin{align*}
f_x & = C_{1x}\cos(k_x x) + C_{2x}\sin(k_x x) \, , \\
f_y & = C_{1y}\cos(k_y y) + C_{2y}\sin(k_y y) \, ,\\
f_z & = C_{1z}\cos(k_z z) + C_{2z}\sin(k_z z) \, . 
\end{align*}
Using the initial condition that the boxes must zero at both the beginning and end of the box, then we must choose the sine solution and let the $k$'s be multiples of $\pi$; that means
\begin{align*}
f_x & = C_{2x}\sin(\frac{n_x \pi}{l} x) \, ,& f_y & = C_{2y}\sin(\frac{n_y \pi}{l} y) \, , & f_z & = C_{2z}\sin(\frac{n_z \pi}{l} z) \, . 
\end{align*}
This means that the $\displaystyle{\alpha = \left(\frac{n_x \pi}{l}\right)^2 + \left(\frac{n_y \pi}{l}\right)^2 + \left(\frac{n_y \pi}{l}\right)^2 = \left(\frac{\pi}{l}\right)^2(n_x^2 + n_y^2 + n_z^2)}$. This gives that $\displaystyle{\omega = \frac{\pi c}{l}\sqrt{n_x^2 + n_y^2 + n_z^2}}$. Taking the product of these four, and combining all the coefficients, gives
\[\boxed{f(x,y,z,t) = A \cos(\omega t)\sin(\frac{n_x \pi}{l} x)\sin(\frac{n_y \pi}{l} y)\sin(\frac{n_z \pi}{l} z)} \quad \text{where } \omega = \frac{\pi c}{l}\sqrt{n_x^2 + n_y^2 + n_z^2} \, .\]

\item Taking the analogy that the wavenumbers are on the surface of a sphere, we get that the ``surface area" is
\[2\left(\frac{1}{8}4\pi n^2\right)\dd{n} = \pi n^2 \dd{n} \, .\]
We are given that 
\[E_n = 2\frac{\pi\hbar c}{l}n = \hbar \omega \rightarrow n = \frac{\omega l}{\pi c} \text{ and } \dd{n} = \frac{l}{\pi c}\dd{\omega} \, .\]
We can then say
\begin{align*}
\dd{E} & = \hbar \omega \pi n^2 \dd{n}\\
& = \hbar \omega \pi \left(\frac{\omega l}{\pi c}\right)^2 \left(\frac{l}{\pi c}\right) \dd{\omega}\\
\frac{\dd{E}}{l^3} & = \underbrace{\frac{\hbar}{\pi^2}\left(\frac{\omega}{c}\right)^3}_{\rho(\omega)} \dd{\omega} \, .
\end{align*}

\item Substituting into 11.54 gives $\displaystyle{R_{b\rightarrow a} = \frac{\omega^3\abs{\mathcal{P}}^2}{3\pi \varepsilon_0 \hbar c^3}}$, which is exactly what we got in 11.63.

\end{enumerate}

%--------------------------------------------------------------------------------
\newpage

\section*{Griffiths 11.13}
A lot of the analysis for this one mirrors what we have already done in the first question. From that conversation we can copy that $\displaystyle{\bra{1 0 0 } z \ket{2 1 0} = \frac{2^8 a}{3^5 \sqrt{2}}}$. We also noticed that the equations were even in $x$ and $y$ for all of them except $\ket{2 1 \pm 1}$. This is the only matrix element that we still need to calculate.
\begin{align*}
\bra{1 0 0} x \ket{2 1 \pm 1} & = \int \frac{1}{\sqrt{\pi a^3}}\frac{\mp 1}{\sqrt{64\pi a^3}}e^{-r/a}\frac{r}{a}e^{-r/2a}\sin(\theta)e^{\pm i \phi} x\dd{V}\\
& = \frac{\mp 1}{8\pi a^4}\left[4!\left(\frac{2a}{3}\right)^5\right]\frac{4\pi}{3} = \mp \frac{2^7}{3^5}a \, ,\\
\bra{1 0 0} y \ket{2 1 \pm 1} & = \int \frac{1}{\sqrt{\pi a^3}}\frac{\mp 1}{\sqrt{64\pi a^3}}e^{-r/a}\frac{r}{a}e^{-r/2a}\sin(\theta)e^{\pm i \phi} y\dd{V}\\
& = \frac{\mp 1}{8\pi a^4}\left[4!\left(\frac{2a}{3}\right)^5\right]\frac{4\pi}{3}(\pm i) = -i \frac{2^7}{3^5}a \, .
\end{align*}
Putting these all together give that 
\begin{align*}
\bra{1 0 0}\vb{r}\ket{2 0 0} & = 0 \, , & \bra{1 0 0}\vb{r}\ket{2 1 0} & =\frac{2^8 a}{3^5 \sqrt{2}}\hat{z} & \bra{1 0 0}\vb{r}\ket{2 1 \pm 1} & = \frac{2^7}{3^5}a(\mp \hat{x} - i \hat{y}) \, , \\
\bra{1 0 0}\vb{r}\ket{2 0 0} & \rightarrow \abs{\mathcal{P}}^2 = 0 & \bra{1 0 0}\vb{r}\ket{2 1 0} & \rightarrow \abs{\mathcal{P}}^2 = (qa)^2\frac{2^{15}}{3^{10}}  & \bra{1 0 0}\vb{r}\ket{2 0 0} & \rightarrow \abs{\mathcal{P}}^2 = (qa)^2\frac{2^{15}}{3^{10}} \, .
\end{align*}
We are given that $\displaystyle{A = \frac{\omega^3 \abs{\mathcal{P}}^2}{3\pi \varepsilon_0 \hbar c^3}}$, and $\displaystyle{\omega = \frac{E_2-E_1}{\hbar} = -\frac{3}{4\hbar}E_1}$. So, for all the states where $\mathcal{P}\neq 0$:
\begin{align*}
A & = -\frac{3^3}{2^6}\frac{E_1^3}{\hbar^3}\frac{(ea)^2 2^{15}}{3^{10}}\frac{1}{3\pi\varepsilon_0 \hbar c^3} \rightarrow 6.27\times 10^8 \text{s}^{-1} \, ,
\end{align*}
\[\boxed{\tau = \frac{1}{A} \approx 1.60\times 10^{-9} \text{s}} \, .\qquad \checkmark\]
Obviously, the $\ket{2 0 0 }$ state never has the opportunity to transition, so the lifetime is $\tau \rightarrow \infty$. 

\end{document}
