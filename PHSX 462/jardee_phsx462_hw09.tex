 \documentclass[11pt]{article}

\usepackage[english]{babel}
\usepackage[margin=1in]{geometry}

% Math/Greek packages
\usepackage{amssymb,amsmath,amsthm, mathtools} 
\usepackage{algorithm, algorithmic}
\usepackage{upgreek, siunitx}
\usepackage{setspace}

% Graphics/Presentation packages
\usepackage{multirow}
\usepackage{graphicx}
\usepackage{cancel}
\usepackage{tabulary, enumitem, array}
\usepackage{xparse,mleftright,tikz}
\usepackage{physics}

% Misc packages
\usepackage{fancyhdr}


\usepackage[export]{adjustbox}

\usepackage{esint}

\sisetup{locale=US,group-separator = {,}}
\usepackage[colorlinks=true, allcolors=blue]{hyperref}


\begin{document}

\title{PHSX 462: HW09}
\author{William Jardee}
\maketitle


\section*{Griffiths 11.2}
Let's start here by simply copying the wavefunctions that we will need:
\begin{align*}
\psi_{100} & = \frac{1}{\sqrt{\pi a^3}}e^{-r/a} \, ,\\
\psi_{200} & = \frac{1}{\sqrt{8\pi a^3}}\left(1 - \frac{r}{2a}\right)e^{-r/2a} \, ,\\
\psi_{210} & = \frac{1}{\sqrt{32\pi a^3}}\frac{r}{a} e^{-r/2a}\cos(\theta) \, ,\\
\psi_{21\pm 1} & = \mp \frac{1}{\sqrt{64 \pi a^3}}\frac{r}{a} e^{-r/2a}\sin(\theta)e^{\pm i\phi} \, .
\end{align*}
Since $r$ has the same degree as $z$ ($r^2 = x^2 + y^2 + z^2$), then we can just do some analysis of each as even or odd, and eliminate each on that. For each of the $H^\prime_{ii}$, they are even in $z$; thus, each is zero. The same holds for each of the combinations of $\ket{200}, \ket{211}, \ket{21-1}$ with $\ket{100}$. However, the combination of $\ket{210}$ with $\ket{100}$ is not, so let us look at that:
\begin{align*}
H^\prime_{100, 210} & = eE\frac{1}{\sqrt{\pi a^3}}\frac{1}{\sqrt{32 \pi a^3}}\frac{1}{a}\int re^{-r/a} e^{-r/2a}\cos(\theta) z r^2 \sin(\theta) \dd{r}\dd{\theta}\dd{\phi}\\
& = eE \frac{1}{a^4\sqrt{8}\pi}\int^\infty_0 r^4 e^{-3r/2a} \dd{r} \int_0^\pi \cos[2](\theta) \dd{\theta} \int^{2\pi}_0 \dd{\phi}\\
& = \frac{eE}{2\sqrt{2}a^4}4!\left(\frac{2a}{3}\right)^5\frac{2}{3}\\
& = \left(\frac{2^{17/2}}{3^5}\right)eEa \, .
\end{align*}
So, every matrix element is zero except 
\[\boxed{H^\prime_{100, 210} = \left(\frac{2^{17/2}}{3^5}\right)eEa} \, .\]

%--------------------------------------------------------------------------------
\newpage

\section*{Griffiths 11.8}
\begin{enumerate}[label=\alph*)]
\item According to first order perturbation theory, we are looking at
\begin{align*}
c_a^{(1)}(t) & = 1\\
c_b^{(1)}(t) & = -\frac{i}{\hbar} \int_{t_0}^t H^\prime_{ba}(t^\prime)e^{i\omega_0 t^\prime}\dd{t^\prime}\\
& = - \frac{i}{\hbar}\int_{-\infty}^t\frac{\alpha}{\sqrt{\pi}\tau} e^{-(t^\prime/\tau)^2}e^{i\omega t^\prime} \dd{t^\prime}\\
& = -\frac{i}{\hbar}\frac{\alpha}{\sqrt{\pi}\tau}\int_{-\infty}^t e^{-\frac{1}{\tau^2}\left(t^\prime - \frac{i\omega\tau^2}{2}\right)^2 - \frac{\omega^2 \tau^2}{4}}\, 
\end{align*}
taking into consideration that we will be taking $t \rightarrow \infty$
\begin{align*}
c_b^{(1)}(t) & = - \frac{i\alpha}{\hbar \sqrt{\pi}}e^{-(\omega_0 \tau/2)^2}\sqrt{\pi}\\
& = - \frac{i\alpha}{\hbar}e^{-(\omega_0 \tau/2)^2} \, .
\end{align*}
The probability of the transition will then be 
\[\boxed{P_{a\rightarrow b} = \left(\frac{\alpha}{\hbar}\right)^2 e^{-(\omega_0 \tau)^2/2}} \, .\]

\item Applying a slight modification on the previous part:
\[c_b^{(1)} = -\frac{i}{\hbar}\int_{-\infty}^\infty \alpha \delta(t)e^{i\omega t}\dd{t} = -\frac{i\alpha}{\hbar} \, .\]
This gives a probability of $(\alpha/\hbar)^2$, which is the first order term of the solution of 11.4 ($\sin[2](\alpha/\hbar)$).

\item The probability gets killed off, as $e^{-(\omega \tau)^2/2}\rightarrow 0$. Consequently, we see no transition of states and the hinting at an adiabatic-type system.

\end{enumerate}

%--------------------------------------------------------------------------------
\newpage

\section*{Griffiths 11.9}
\textit{This is post homework Will typing this up; this question was a pain in the butt while doing it, but actually super fun on the other side. Since I am straight exhausted, I am going to be skipping some steps to save my sanity...}
\begin{enumerate}[label=\alph*)]
\item Plugging in what we are given
\begin{align*}
\dot{c}_a & = -\frac{i}{\hbar}H^\prime_{ab}e^{-i\omega_0 t}c_b  = -\frac{i}{2\hbar}V_{ab}e^{i(\omega - \omega_0)t}c_b\\
\dot{c}_b & = -\frac{i}{\hbar}H^\prime_{ba}e^{i\omega_0 t}c_a  = -\frac{i}{2\hbar}V_{ab}e^{i(\omega_0 - \omega)t}c_a \, .
\end{align*}
We must go down before we can go up, so let us differentiate $\dot{c_a}$:
\begin{align*}
\ddot{c}_a &= -\frac{i}{2\hbar}V_{ab}\left[i(\omega-\omega_0)e^{i(\omega-\omega_0)t}c_b + e^{i(\omega-\omega_0)t}\dot{c}_b\right]\\
& = - \frac{i}{2\hbar}V_{ab}\left[i(\omega-\omega_0)\frac{i2\hbar}{V_{ab}}\dot{c}_a + e^{i(\omega-\omega_0)t}\left(-\frac{i}{2\hbar}\right)V_{ba}e^{i(\omega_0-\omega)t}c_a\right]\\
& = i(\omega-\omega_0)\dot{c}_a - \frac{\abs{V_{ab}}^2}{(2\hbar)^2}c_a \, . 
\end{align*}
Rewriting this gives
\[\ddot{c}_a + i(\omega_0-\omega)\dot{c}_a +\frac{\abs{V_{ab}}^2}{(2\hbar)^2}c_a = 0 \, . \]
This is a differential equation, that if we propose an exponential solution in the form of $e^{\lambda t}$, gives
\[\lambda^2 + i(\omega_0-\omega)\lambda +\frac{\abs{V_{ab}}^2}{(2\hbar)^2} = 0\]
and, via the quadratic formula
\[\lambda = -\frac{1}{2} \left[i(\omega-\omega_0)\pm \sqrt{-(\omega_0-\omega)^2 - \frac{4\abs{V_{ab}}^2}{(2\hbar)^2}}\right] = -i\left[\frac{(\omega - \omega_0)}{2} \pm \omega_r\right] \, .\]
the general equation for $c_a$ is then
\[c_a(t) = Ae^{-i((\omega - \omega_0)/2 + \omega_r)t} + Be^{-i((\omega - \omega_0)/2 - \omega_r)t} \, .\]
Using the initial conditions gives $A+B = 1$.

By a very similar derivation, the general solution for $c_b$ is 
\[c_a(t) = Ce^{-i((\omega - \omega_0)/2 + \omega_r)t} + De^{-i((\omega - \omega_0)/2 - \omega_r)t} \, .\]
Using the initial conditions here gives $C+D = 0$.

To solve for the coefficients in the $c_b$ equation, let us take the derivative: 
\begin{align*}
\dot{c}_b & = -i\left[\frac{(\omega - \omega_0)}{2} + \omega_r\right]Ce^{-i\left[\frac{(\omega - \omega_0)}{2} + \omega_r\right]} -i\left[\frac{(\omega - \omega_0)}{2} - \omega_r\right]De^{-i\left[\frac{(\omega - \omega_0)}{2} - \omega_r\right]}\\
-\frac{i}{2\hbar}V_{ba}e^{i(\omega_0-\omega)t}c_a & = \qquad " \quad "\\
c_a & = \left[\frac{2\hbar}{V_{ba}}\right]\left[\left(\frac{(\omega-omega_0)}{2} + \omega_r\right)Ce^{-i\omega_r t} + \left(\frac{(\omega-omega_0)}{2} - \omega_r\right)De^{i\omega_r t}\right]\\
1 & = \left[\frac{2\hbar}{V_{ba}}\right]\left[\left(\frac{(\omega-omega_0)}{2} + \omega_r\right)C + \left(\frac{(\omega-omega_0)}{2} - \omega_r\right)D\right]\\
1 & = \left[\frac{2\hbar}{V_{ba}}\right][C-D]\omega_r
\end{align*}
\[C - D = \frac{V_{ba}}{2\hbar \omega_r} \, .\]
Combining this with an initial condition gives
\[2C = \frac{V_{ba}}{2\hbar \omega_r} \rightarrow C = \frac{V_{ba}}{4\hbar \omega_r} \, \text{ and } \, D = -\frac{V_{ba}}{4\hbar \omega_r} \, .\]
At this point I am realizing my flaw in keeping everything in complex exponentials. On my scratch work I worked out what this is, specifically when I plugged in $C$ and $D$; 
\[c_b(t)  = -\frac{i}{2\hbar \omega_r}V_{ba} \left[e^{i((\omega-\omega_0)/2)t}\right]\sin(\omega_r t) \, .\]

A very similar process will need to be done with $c_a$. I will skip typing it all up, but we get
\begin{align*}
c_b & = \frac{V_{ab}}{2\hbar}\left[\left(\frac{\omega_0 - \omega}{2} + \omega_r\right)Ae^{-i\omega_r t} + \left(\frac{\omega_0 - \omega}{2} - \omega_r\right)Be^{i\omega_r t}\right]\\
0 & = \frac{V_{ab}}{2\hbar}\left[\left(\frac{\omega_0 - \omega}{2} + \omega_r\right)A + \left(\frac{\omega_0 - \omega}{2} - \omega_r\right)B\right]\\
0 & = \frac{\omega_0-\omega}{2}+ \omega_r[A-B] \, .
\end{align*}
Combining this equation with the initial values from before we get
\[2A = 1 + \frac{\omega-\omega_0}{2\omega_r} \rightarrow A = \frac{1}{2} + \frac{\omega-\omega}{4\omega_r} \, \text{ and } \, B = \frac{1}{2} - \frac{\omega-\omega}{4\omega_r} \, .\]
Plugging these into $c_a$, and doing some simplifying,
\[c_a(t) = e^{-i((\omega_0 - \omega)/2)t}\left[\cos(\omega_r t) + \frac{i(\omega_0 - \omega)}{2\omega_r}\sin(\omega_r t)\right] \, .\]

Finally, the two constants are 
\[\boxed{\mqty{c_a(t) = e^{-i((\omega_0 - \omega)/2)t}\left[\cos(\omega_r t) + \frac{i(\omega_0 - \omega)}{2\omega_r}\sin(\omega_r t)\right] \\ c_b(t)  = -\frac{i}{2\hbar \omega_r}V_{ba} \left[e^{i((\omega-\omega_0)/2)t}\right]\sin(\omega_r t)}}\]


\end{enumerate}

%--------------------------------------------------------------------------------
\newpage

\section*{Griffiths 11.11}


%--------------------------------------------------------------------------------
\newpage

\section*{Griffiths 11.13}


\end{document}
