\documentclass[11pt]{article}

\usepackage[english]{babel}
\usepackage[margin=0.8in]{geometry}

% Math/Greek packages
\usepackage{amssymb,amsmath,amsthm, mathtools} 
\usepackage{algorithm, algorithmic}
\usepackage{upgreek, siunitx}
\usepackage{setspace}

% Graphics/Presentation packages
\usepackage{multirow}
\usepackage{graphicx}
\usepackage{cancel}
\usepackage{tabulary, enumitem, array}
\usepackage{xparse,mleftright,tikz}
\usepackage{physics}

% Misc packages
\usepackage{fancyhdr}


\usepackage[export]{adjustbox}

\usepackage{esint}

\sisetup{locale=US,group-separator = {,}}
\usepackage[colorlinks=true, allcolors=blue]{hyperref}


% Box function - update this as more sophisticated solutions are found
\newcommand\mybox[2][]{\tikz[overlay]\node[fill=blue!20,inner sep=2pt, anchor=text, rectangle, rounded corners=1mm,#1] {#2};\phantom{#2}}
\renewcommand{\arraystretch}{1.2}

% General macro declarations


\makeatletter
\let\oldabs\abs
\def\abs{\@ifstar{\oldabs}{\oldabs*}}
%
\let\oldnorm\norm
\def\norm{\@ifstar{\oldnorm}{\oldnorm*}}
\makeatother

\begin{document}

\title{PHSX 499: Selected Paper Outline v.2}
\author{William Jardee}
\maketitle

This is an analysis of the work published in 2020 by Thibaut Vidal and Maximilian Schiffer, \textbf{Born-Again Tree Ensembles}. I go through section by section and highlight what I believe is the ``topic sentence" of the paragraph. In some cases there are a couple topic sentences for a section, it should be clear why I chose multiple. The paper can be found at citation \cite{pmlr-v119-vidal20a}.

%--------------------------------------------------------

\section{Introduction}	
``Tree ensembles constitute a core technique for prediction and classification tasks."\\
``Currently, there exists a trade-off between the interpretability and the performance of tree (ensemble) classifiers." 

\subsection{State of the Art}
\subsubsection*{Thinning tree ensembles}
``Thinning tree ensembles has been studied from different perspectives and divides in two different streams, $i)$ classically thinning a tree ensemble by removing some weak learners from the original ensemble and $ii)$ replacing a tree ensemble by a simpler classifier, e.g., a single decision tree."\\
``In the field of neural networks, related studies were done on \textit{model compression}."
\subsubsection*{Decision trees}
``since the 1990's, some works focused on constructing decision trees based on mathematical programming techniques."
\subsection{Contributions}
``With this work, we revive the concept of born-again tree ensembles and aim to construct a single -minimum-size-tree that faithfully reproduces the decision function of the original tree ensemble."

%--------------------------------------------------------

\section{Fundamentals}
``In this section, we introduce some fundamental definitions."

%--------------------------------------------------------

\section{Methodology}
``In thiss section, we introduce a dynamic programming (DP) algorithm which optimally solves Problem 1 for many data sets of practical interest."

%--------------------------------------------------------

\section{Computational Experiments}
``The goal of our computational experiments is fourfold: ..." \textit{The authors then go on to outline the four experiments that are conducted.}
\subsection{Data Preparation}
``We focus on a set of six datasets from the UCI machine learning repository [UCI] and from previous work by ..."
\subsection{Computational Effort}
``In a firsts analysis, we evaluate the computation time of Algorithm 1 for different data sets and size metrics."\\
``In our second analysis, we focus on the FICO case and randomly extract subsets of samples and features to produce smaller data sets."\\
``We observe that the computational time fo the DP algorithm is strongly driven by the number of features, with an exponential growth relative to their parameter."
\subsection{Complexity of the Born-Again Trees}
``We now analyze the depth and number of leaves of the born-again trees for different objective functions and datasets in Table 2."
\subsection{Post-Pruned Born-Again Trees}
``To circumvent this issue, we suggest to apply a simple post-pruning step to eliminate inexpressive tree sub-regions."
\subsection{Heuristic bone-Again Trees}
``Accordingly, we take a first step towards scalable heuristic algorithms in the following."

%--------------------------------------------------------

\section{Conclusions}
``In this paper, we introduced an efficient algorithm to transform a random forest into a single, smallest possible, decision tree."\\
``As a perspective for future work, we recommend to progress further on solution techniques for the born-again tree ensembles problem, proposing new optimal algorithms to effectively handle larger datasets as well as fast and accurate heuristics."


\vskip 0.2in
\bibliographystyle{unsrt}

\bibliography{phsx499_selectedpaperoutline}


\end{document}
