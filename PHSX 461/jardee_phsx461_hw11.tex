\documentclass[12pt]{article}

\usepackage[english]{babel}

% Math/Greek packages
\usepackage{amssymb,amsmath,amsthm, mathtools} 
\usepackage{algorithm, algorithmic}
\usepackage{upgreek, siunitx}

% Graphics/Presentation packages
\usepackage{geometry, graphicx}
\usepackage{tabulary, enumitem, array}
\usepackage{xparse,mleftright,tikz}
\usepackage{physics}

% Misc packages
\usepackage{fancyhdr}


\usepackage[export]{adjustbox}

\usepackage{esint}

\sisetup{locale=US,group-separator = {,}}
\usepackage[colorlinks=true, allcolors=blue]{hyperref}


% Box function - update this as more sophisticated solutions are found
\newcommand\mybox[2][]{\tikz[overlay]\node[fill=blue!20,inner sep=2pt, anchor=text, rectangle, rounded corners=1mm,#1] {#2};\phantom{#2}}
\renewcommand{\arraystretch}{1.2}

% General macro declarations


\makeatletter
\let\oldabs\abs
\def\abs{\@ifstar{\oldabs}{\oldabs*}}
%
\let\oldnorm\norm
\def\norm{\@ifstar{\oldnorm}{\oldnorm*}}
\makeatother

\begin{document}

\title{PHSX 461: HW11}
\author{William Jardee}
\maketitle

\section*{Griffiths 4.12}
{\sl Work out the radial wave functions of $R_{31}$, using the recursion formula. Don't bother to normalize them.}

\begin{equation}
	\tag{Equation 4.76}
	c_{j+1} = \frac{2(j+l+1-n)}{(j+1)(j+2l+2)}c_j
	\label{eq:4.76}
\end{equation}

\section*{Griffiths 4.15}
\begin{enumerate}[label=\alph*)]
\item {\sl Find $\ev{r}$ and $\ev{r^2}$ for an electron in the ground state of hydrogen. Express your answers in terms of the Bohr radius.}

\item {\sl Find $\ev{x}$ and $\ev{x^2}$ for an electron in the ground state of hydrogen.}

\item {\sl Find $\ev{x^2}$ in the state $n=2, l=1, m=1$.}

\end{enumerate}

\section*{Griffiths 4.16}
{\sl What is the most probably value of r, in the ground state of hydrogen?}

\section*{Griffiths 4.27}
{\sl Two particles $($masses $m_1$ and $m_2)$ are attached to the ends of a massless rigid rod of length $a$. the system is free to rotate in three dimensions about the $($fixed$)$ center of mass.}
\begin{enumerate}[label=\alph*)]
\item {\sl Show that the allowed energies of this rigid rotor are}
\[E_n = \frac{\hbar^2}{2I}n(n+1) \quad (n = 0,1,2,...) \quad \text{where } I = \frac{m_1 m_2}{m_1 + m_2}a^2\]
{\sl is the moment of inertia of the system}

\item {\sl What are the normalized eigenfunctions for this system? $($Let $\theta$ and $\phi$ define the orientation of the rotor axis.$)$ What is the degeneracy of the $n$th energy level?}

\item {\sl What spectrum would you expect for this system?}

\item {\sl According to the figure given in the text, what is the frequency separation between adjacent lines? Look up the masses of $^{12}C$ and $^{16}O$, and from $m_1$, $m_2$, and $\delta \nu$ determine the distance between the atoms.}

\end{enumerate}


\section*{Griffiths 4.64}
{\sl the electron in a hydrogen atom occupies the combined spin and position state}
\[R_{21}(\sqrt{1/3}Y^0_1 \chi_+ + \sqrt{2/3}Y^1_1 \chi_-)\]
\begin{enumerate}[label=\alph*)]
\item {\sl If you measured the orbital angular momentum squared, what values might you get, and what is the probability of each? }

\item {\sl Same for the $z$ component of angular momentum.}

\item {\sl Same for the spin angular momentum squared.}

\item {\sl Same for the $z$ component of spin angular momentum.}

\item {\sl Same for the energy of the electron.}
\end{enumerate}


\section*{Griffiths 4.30}
{\sl An electron is in the spin state}
\[\chi = A \mqty(3i & 4)\]
\begin{enumerate}[label=\alph*)]
\item {\sl Determine the normalization constant $A$}

\item {\sl Find the expectation values of $S_x$, $S_y$, and $S_z$}

\item {\sl Find the ``uncertainties" $\sigma_{S_x}$, $\sigma_{S_y}$, and $\sigma_{S_z}$}

\item {\sl Confirm that your results are consistent with all three uncertainty principles.}
\end{enumerate}

\end{document}