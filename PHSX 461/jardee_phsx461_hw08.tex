\documentclass[12pt]{article}

\usepackage[english]{babel}

% Math/Greek packages
\usepackage{amssymb,amsmath,amsthm, mathtools} 
\usepackage{algorithm, algorithmic}
\usepackage{upgreek, siunitx}

% Graphics/Presentation packages
\usepackage{geometry, graphicx}
\usepackage{tabulary, enumitem, array}
\usepackage{xparse,mleftright,tikz}
\usepackage{physics}

% Misc packages
\usepackage{fancyhdr}


\usepackage[export]{adjustbox}

\usepackage{esint}

\sisetup{locale=US,group-separator = {,}}
\usepackage[colorlinks=true, allcolors=blue]{hyperref}


% Box function - update this as more sophisticated solutions are found
\newcommand\mybox[2][]{\tikz[overlay]\node[fill=blue!20,inner sep=2pt, anchor=text, rectangle, rounded corners=1mm,#1] {#2};\phantom{#2}}
\renewcommand{\arraystretch}{1.2}

% General macro declarations


\makeatletter
\let\oldabs\abs
\def\abs{\@ifstar{\oldabs}{\oldabs*}}
%
\let\oldnorm\norm
\def\norm{\@ifstar{\oldnorm}{\oldnorm*}}
\makeatother

\begin{document}

\title{PHSX 461: HW07}
\author{William Jardee}
\maketitle

\section*{3.7}
\begin{enumerate}[label=\alph*)]
\item \emph{Suppose that $f(x)$ and $g(x)$ are two eigenfunctions of an operator $\hat{Q}$, with the same eigenvalue $q$. Show than any linear combination of $f$ and $g$ is itself an eigenfunction of $\hat{Q}$, with eigenvalue $q$.}\bigskip

\item \emph{Check that $f(x) = \text{exp} (x)$ and $g(x) = \text{exp} (-x)$ are eigenfunctions of the operator $d^2/dx^2$, with the same eigenvalue. Construct two linear combinations of $f$ and $g$ that are orthogonal eigenfunctions on the interval $(-1,1)$.}
\end{enumerate}

%---------------------------------------------------------------
\newpage

\section*{3.9}
\begin{enumerate}[label=\alph*)]
\item \emph{Cite a Hamiltonian from Chapter 2 (other than the harmonic oscillator) that has only a discrete spectrum.}\bigskip

\item \emph{Cite a Hamiltonian from Chapter 2 (other than the free particle) that has only a continuous spectrum.}\bigskip

\item \emph{Cite a Hamiltonian from Chapter 2 (other than the finite square well) that has both a discrete and a continuous part to its spectrum.}
\end{enumerate}

%---------------------------------------------------------------
\newpage

\section*{3.13}
\emph{Show that}
\[\langle x \rangle = \int \Phi^* \qty(i\hbar\pdv{p})\Phi \dd{p}\]
\emph{Notice that $x\text{exp}(ipx/\hbar) = -i\hbar (\pdv*{p})\text{exp} (ipx/\hbar)$, and use Equation 2.147. In momentum space, then, the position operator is $i\hbar \pdv*{p}$. More generally,}\bigskip

%---------------------------------------------------------------
\newpage

\section*{3.26}
\emph{Consider a three-dimensional vector space spanned by an orthonormal basis $\ket{1},\,\ket{2},\,\ket{3}$. Kets $\ket{\alpha}$ and $\ket{\beta}$ are given by}
 \[\ket{\alpha} = i\ket{1} -2\ket{2}-i\ket{3}, \qquad \ket{\beta} = i\ket{1} + 2\ket{3}\]
\begin{enumerate}[label=\alph*)]
\item \emph{Construcct $\bra{\alpha}$ and $\bra{\beta}$ (in terms of the dual basis $\bra{1},\,\bra{2},\,\bra{3}$).}\bigskip

\item \emph{Find $\braket{\alpha}{\beta}$ and $\braket{\beta}{\alpha}$, and confirm that $\braket{\beta}{\alpha} = \braket{\alpha}{\beta}^*$.}\bigskip

\item \emph{Final all nine matrix elements fo the operator $\hat{A} = \ket{\alpha}\bra{\beta}$, in this basis, and construct the matrix $A$. Is it hermitian?}
\end{enumerate}

%---------------------------------------------------------------
\newpage

\section*{Question 5.}
\emph{Prove that the momentum operator, $\hat{p}$ is Hermitian.}\\
\emph{\textbf{Hint:} you will need to assume that any functions you use are normalizable. You may also use the results from the previous homework assignment.}\bigskip

%---------------------------------------------------------------
\newpage

\section*{3.33}
\emph{An operator $\hat{A}$, representing observable $A$, has two (normalized) eigenstates $\psi_1$ and $\psi_2$, with eigenvalues $a_1$ and $a_2$, respectively. Operator $\hat{B}$, representing observable $B$, has two (normalized) eigenstates $\phi_1$ and $\phi_2$, with eigenvalues $b_1$ and $b_2$. The eigenstates are related by}
\[\psi_1 = (3\phi_1 + 4\phi_2)/5, \qquad \psi_2 = (4\phi_1 - 3\phi_2)/5\]
\begin{enumerate}[label=\alph*)]
\item \emph{Observable $A$ is measured, and the value $a_1$ is obtained. What is the state of the system (immediately after this measurement?}\bigskip

\item \emph{If $B$ is now measure, what are the possible results, and what are their probabilities?}\bigskip

\item \emph{Right after the measurement of $B$, $A$ is measured again. What is the probability of getting $a_1$? (Note that the answer would be quite different if I had told you the outcome of the $B$ measurement.}
\end{enumerate}
\end{document}