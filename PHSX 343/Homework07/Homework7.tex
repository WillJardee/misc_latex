\documentclass[11pt]{article}

\usepackage{amssymb,amsmath}
\usepackage{times,psfrag,epsf,epsfig,graphics,graphicx,caption}
\usepackage{enumitem}
\usepackage{algorithm}
\usepackage{algorithmic}

\begin{document}
\date{}

\title{PHSX 343: Assignment 7}

\author{William Jardee}

\maketitle


\section*{Problem 1}
    \begin{enumerate}[label=\alph*)]
        Using Lorentz Translations, just replacing the variable with deltas:
        \begin{center}
            $\Delta x' = \gamma (\Delta x + v\Delta t)$ \quad $\Delta t'= \gamma \Big(\Delta t = \frac{v}{c}\Delta x\Big)$
        \end{center}
    \item
    \begin{itemize}
        \item 
        If the events are simultaneous, but not collocated, $\Detla t' = 0$ and thus:
        \[\Delta t' = 0 = \gamma \Big(\Delta t + \frac{v}{c}\Delta x\Big)\]
        \[\Delta t' = \frac{c}{v} \Detla t > c \Delta t\]
        \item
        If $\Detla x < c\Delta t$, as would be true if $v>c$, then $\Delta t'$ could be negative. This means that the event ran in reverse and would violate the first law of thermodynamics. As we know from the first postulate of relativity, that the laws of physics are consistent between any frame.
    \end{itemize}

    \item
    \begin{itemize}
        \item 
        If $\Delta  x \leq c \Delta t$,  then:
        \[\Delta t' \geq \gamma \Big( c \Delta x +\frac{v}{c}\Delta x \Big)= \gamma \Big(c +\frac{v}{c}\Big)\Delta x\]
        So, if $v>0$, then $\Delta x > 0$, and consequently $\Delta t' \geq 0$
        \item
        If the object is moving in the negative x direction, when $\frac{v}{c}\Delta x < \Delta t$, then the $\Delta t'< 0$. Then it could be a logical jump that Event B caused Event A if the sign of the change was the only thing that mattered. This is an obvious contradiction to what the other frame saw, and the law of physics have to stay consitat (first postulate of relativity), so this cannot be the case.
        
        
    \end{itemize}
    \item
    \begin{itemize}
        \item 
        \[\Delta t' = \gamma \Big(\Delta t + \frac{v \alpha}{c}\Delta x \Big)\Delta t = \gamma \Big(1+ \frac{v \alpha}{c}\Big) \Delta t<0\]
        \[\frac{v \alpha}{c} < -1 \rightarrow v<-\frac{c}{\alpha}\]
        We can restrict our solution to this one answer because we know the velocity moves in the $+x direction$
        \item
        We know from the Lorentz Transforms that $t'=\gamma t$. So, if $v>c$ then $\gamma$ is imaginary and we get an imaginary time. We cannot have an imaginary time for a real particle in this setup, so if we take that a particle can travel faster than the speed of light then we get a time that doesn't make sense with the laws of physics. So we get a contradiction with the first law of relativity.
    \end{itemize}
    
    \end{enumerate}

\section*{Problem 2}

\begin{enumerate}[label=\alph*)]
    \item The Lorentz Transform will use the frame of A as un-primed, since both B and C are moving in the -x direction for it. So the equation for velocity gives:
        \[V_x = \frac{V_x' + \beta c}{1 + \beta \frac{V_x'}{c}}\]
        For $V_B = \frac{V_B'+\beta c}{1+\beta\frac{V_B'}{c}} \rightarrow  \frac{0.8c+\sqrt{1-0.8^2}c}{1+\sqrt{1-0.8^2}0.8} = 0.946c$\\
        $V_C = \frac{0+\beta c}{1} = 0.8c$
    \item
        \parbox{0}{ \includegraphics[width = 400pt]{Homework07/Homework7.PNG}}\\
        Estimating the values, $\Delta x_B = -1.9 \gamma$ and $\Delta t_B = 1.95 \gamma$, so $v_B = 0.974c$. $\Delta x_C = 1 \gamma$ and $\Delta t_C = 1.2 \gamma$, so $v_C = 0.83c$. These numbers are moderately close to those given in part a. 
    \item
        For the lab's frame, we get that:
        \[\Delta p_l = (mv_{iA} + mv_{iB})-mv_{fc} = m(0.8c-0.8c) - m(0) = 0\]
        So the momentum is conserved. \\
        Now for A's frame:
        \[\Delta p_A = mv_{iA}+ mv_{iB} - mv_{fC}= m(-0.946-0.8)c \neq 0\]
        So the momentum is not conserved in A's frame.

\end{enumerate}


\end{document}
