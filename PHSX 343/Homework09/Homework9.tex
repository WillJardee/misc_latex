\documentclass[11pt]{article}

\usepackage{amssymb,amsmath}
\usepackage{times,psfrag,epsf,epsfig,graphics,graphicx,caption}
\usepackage{enumitem}
\usepackage{algorithm}
\usepackage{algorithmic}

\begin{document}
\date{}

\title{PHSX 343: Assignment 9}

\author{William Jardee}

\maketitle


\section*{Problem 1}
    \begin{enumerate}[label=\alph*)]
        \item 
            With conservation of momentum, not all the energy of the photon will go into the mass of the particle. $0.01Mc^2$ will go into the mass, but another batch will go in to bring the momentum of the particle up to the initial momentum of the photon. 
        \item
            \[E_i = E_{Mi} + E_p = Mc^2 + P_p c\]
            \[E_f = E_{Mf} = \sqrt{(P_pc)^2 + (1.01Mc^2)^2}\]
            For conservation of momentum:
            \[E_i = E_f \rightarrow (Mc^2) + (P_p c) = \sqrt{(P_p c)^2 + (1.01Mc^2)^2}\]
            \[(Mc^2)^2 + 2(Mc^2)(P_p c) + (P_p c)^2 = (P_p c)^2 + (1.01 Mc^2)^2\]
            \[(1.01^2 -1)(Mc^2)^2 = 2 (mc^2)(P_p c)\]
            \[\frac{1.01^2 -1}{2}Mc^2 = P_p c = E_p = 0.01005 Mc^2\]
        
    \end{enumerate}

\section*{Problem 2}
    \begin{enumerate}[label=\alph*)]
        \item 
        For the conservation of 4-momentum: \\
        For the energy of the photon, $E^2 = (Pc)^2 + (Mc^2)^2 = (Pc)^2$ and for $E = P_tc$.
        \[\begin{bmatrix}
        E_1/c \\ P_x \\ 0 \\ 0
        \end{bmatrix}
        +
        \begin{bmatrix}
        E_2/c \\ -P_x \\ 0 \\ 0
        \end{bmatrix}
        =
        \begin{bmatrix}
        P_p \\ P_p \\ 0 \\ 0
        \end{bmatrix}\]
        But it's evident that $P_x-P_x = 0 = P_p$, so the photon would have no momentum and have no energy, so there would be no photon actually created for a valid formula. 
        \item
        This time the 2nd mass will have no momentum:\\
        \[\begin{bmatrix}
        E_1/c \\ P_x \\ 0 \\ 0
        \end{bmatrix}
        +
        \begin{bmatrix}
        Mc \\ 0 \\ 0 \\ 0
        \end{bmatrix}
        =
        \begin{bmatrix}
        P_p \\ P_p \\ 0 \\ 0
        \end{bmatrix}\]
        \[\frac{E_1}{c}= \frac{\sqrt{(P_xc^2)+ (Mc^2)^2}}{c} = \sqrt{P_x^2 + (Mc)^2}\]
        With conservation of momentum we can say that this $P$ is the same as the momentum of the photon: $P_x = P_p$.
        \[\frac{E_1}{c} + Mc = \sqrt{P_p^2+ (Mc)^2} +Mc = P_p\]
        This is only true when $M = 0$. With that condition, we are not dealing with elementary particles, so this situation is brings up a contradiction and is invalid. 
        \item
        If we make the situation abstract, for any starting energy and momentum (kept in 1D), then:\\
        \[\begin{bmatrix}
        E_1/c \\ P_{x1} \\ 0 \\ 0
        \end{bmatrix}
        +
        \begin{bmatrix}
        E_2/c \\ P_{x2} \\ 0 \\ 0
        \end{bmatrix}
        =
        \begin{bmatrix}
        P_p \\ P_p \\ 0 \\ 0
        \end{bmatrix}\]
        
        If be balance both the $P_t$ and $P_x$ equations: 
        \[\sqrt{P_{x1}^2 + (Mc)^2} + \sqrt{P_{x2}^2 + (Mc)^2} = P_p = P_{x1}+P_{x2}\]
        This is only true when $(Mc)^2 = 0 \rightarrow M=0$. This is the same issue as part c. 
        \item
        Equality is a symmetric operation, so any of these derivations can be done in reverse and it remains illegal in that direction. 
    \end{enumerate}


\end{document}
