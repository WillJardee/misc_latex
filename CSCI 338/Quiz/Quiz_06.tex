%% Document initiation %%
\documentclass[11pt]{article}
\usepackage[utf8]{inputenc}
\usepackage[a4paper, total={6in, 8in}]{geometry}
\setlength{\parskip}{0.5em}


%% Package Declarations %%
\usepackage{amssymb,amsmath, algorithm, algorithmic}
\usepackage{xcolor, times,psfrag,epsf,epsfig,graphics, tabularx, array}
\usepackage{tikz}
\usepackage{multicol, wrapfig, ctable}
\usepackage{fancyhdr, hanging}


%% Common Declarations %%
\newcommand\mybox[2][]{\tikz[overlay]\node[fill=blue!20,inner sep=2pt, anchor=text, rectangle, rounded corners=1mm,#1] {#2};\phantom{#2}}
\renewcommand{\arraystretch}{1.2}

\pagestyle{fancy}
\begin{document}

%% Document Building %%
\graphicspath{{../images/}}

%% Title %%
\title{CSCI 338: Quiz~6~}
\author{William Jardee}
\date{\today}
\maketitle


%% Document Contents %%
\section*{Problem 1}
    In your own language, what is the difference between a problem in P and a problem in NP?\\
    
    Problems in P can be solved in polynomial time, problems in NP can be verified in polynomial time. A lot of the time it is easier to verify something than it is to get the answer in the first place. So, for example, finding the cliches in a graph may not be a polynomial time problem, but we have shown that we can verify it in polynomial time. So it may not be in P, but it is in NP. 


\end{document}