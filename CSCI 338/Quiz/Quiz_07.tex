%% Document initiation %%
\documentclass[11pt]{article}
\usepackage[utf8]{inputenc}
\usepackage[a4paper, total={6in, 8in}]{geometry}
\setlength{\parskip}{0.5em}


%% Package Declarations %%
\usepackage{amssymb,amsmath, algorithm, algorithmic}
\usepackage{xcolor, times,psfrag,epsf,epsfig,graphics, tabularx, array}
\usepackage{tikz}
\usepackage{multicol, wrapfig, ctable}
\usepackage{fancyhdr, hanging}


%% Common Declarations %%
\newcommand\mybox[2][]{\tikz[overlay]\node[fill=blue!20,inner sep=2pt, anchor=text, rectangle, rounded corners=1mm,#1] {#2};\phantom{#2}}
\renewcommand{\arraystretch}{1.2}

\pagestyle{fancy}
\begin{document}

%% Document Building %%
\graphicspath{{../images/}}

%% Title %%
\title{CSCI 338: Quiz~7~}
\author{William Jardee}
\date{\today}
\maketitle


%% Document Contents %%
\section*{Problem 1}
    In your own language, what is the difference between NP and NP-complete?\\
    
    Our understanding in class is that if a problem can be reduced to a NP-complete problem in polynomial time, then it is NP-complete ($B \leq_p A$; I am hazing on the exact order of the wording, but you are trying to do a reduction like this were we know $B$ to be NP-complete). This means nothing to understanding NP-complete, since it is a circular definition. So, reading into it a little bit more, NP-complete is a set of NP problems that can be solved with a brute-force search algorithm, verified quick, and, most notably, be used to solve other NP problems. 
    
    So, the difference between NP and NP-complete is that being NP just requires being able to verify in polynomial time, while being NP-complete requires it to be NP and be able to be reducible to some NP-complete problem (so this circular requirement). 


\end{document}