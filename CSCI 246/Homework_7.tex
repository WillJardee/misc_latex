\documentclass[11pt]{article}

\usepackage{amssymb,amsmath}
\usepackage{times,psfrag,epsf,epsfig,graphics,graphicx}
\usepackage{algorithm}
\usepackage{algorithmic}
\usepackage{xcolor}

\begin{document}
\date{}

\title{CSCI 246: Assignment 7}

\author{William Jardee}

\maketitle

 
\section*{Problem 1.}

\noindent
(1.1) How many odd integers are there from 1,000 to 9,999?
\newline
\noindent
{\bf Answer:}\\
The odd numbers between $1$ and $10,000$ will be the same as the number of evens, so:\\
\[\frac{1}{2}\cdot 10,000 - \frac{1}{2}\cdot 1,000 = 5,000-500 = 4,500\]
\newline
\noindent
(1.2) How many integers from 1,000 to 9,999 have distinct digits?
\newline
\noindent
{\bf Answer:}\\
\[{9 \choose 1}\cdot{9 \choose 1} \cdot {8 \choose 1}\cdot {7 \choose 1} = 9\cdot 9 \cdot 8 \cdot 7 = 4536\]
\newline

\noindent
(1.3) How many odd integers from 1,000 to 9,999 have distinct digits?
\newline
\noindent
{\bf Answer:}\\
\[{5 \choose 1} \cdot {8 \choose 1} \cdot {8 \choose 1} \cdot {7 \choose 1} = 5 \cdot 8\cdot 8\cdot 7 = 2240\]
\newline

\noindent
(1.4) What is the probability that a randomly chosen four-digit integer has distinct digits?
\newline
\noindent
{\bf Answer:}\\
\[P(A) = \frac{N(E)}{S(E)} = \frac{4536}{9000} = 50.4\%\]
\newline

\noindent
(1.5) What is the probability that a randomly chosen four-digit integer has distinct digits and is odd?
\newline
\noindent
{\bf Answer:}\\
\[P(A) = \frac{N(E)}{S(E)} = \frac{2240}{9000} = 24.9\%\]
\newline
\newpage

\section*{Problem 2.}

\noindent
At company $C$, passwords must be from 4-6 symbols long and composed from the
26 uppercase letters of the Roman alphabet, the ten digits 0-9, and the 14
special symbols !, @, \#, \$, \%, $^{\wedge}$, \&, *, (, ), -, +, \{, and \}.
\newline

\noindent
(2.1) How many passwords are possible if repetition of symbols is allowed?
\newline
\newline
\noindent
{\bf Answer:}\\
\[50^4 + 50^5 + 50^6 = 15943750000 = 1.594 \times 10^{10}\]
\newline

\noindent
(2.2) How many passwords contain no repeated symbols?
\newline
\newline
\noindent
{\bf Answer:}\\
\[{50 \choose 4} + {50 \choose 5} + {50 \choose 6} = 18239760\]
\newline

\noindent
(2.3) How many passwords have at least one repeated symbols?
\newline
\newline
\noindent
{\bf Answer:}\\
\[50\cdot {4 \choose 2}\cdot50^2 + 50\cdot {5 \choose 2}\cdot50^3 + 50\cdot {6 \choose 2}\cdot50^4 = 6\cdot 50^3 + 10\cdot 50^4 + 15\cdot 50^5 \]
\[= 4750750000 = 4.750 \times 10^9\]
\newline

\noindent
(2.4) What is the probability that a password chosen at random has at least one repeated symbol? (Round to the nearest tenth of a percent.)
\newline
\newline
\noindent
{\bf Answer:}\\
\[P(A) = \frac{N(E)}{S(E)} = \frac{4750750000}{15943750000} = 30.0\%\]
\newpage


\section*{Problem 3.}

\noindent
(3.1) How many integers from 1 through 1000 are multiples of 2 or multiples of 9?
\newline
\newline
\noindent
{\bf Answer:}
\newline
Since 9 is not an even, the answers will be sets, so: 
\[\frac{1}{2}\cdot 1,000 + \Bigl\lfloor \frac{1}{9} \cdot 1,000 \Bigr\rfloor = 500 + 111 = 611\]
\newline

\noindent
(3.2) Suppose an integer from 1 through 1000 is chosen at random. Use the result of part (3.1) to find the probability that the integer is a multiple of 2 or a multiple of 9. (Enter your probability as a percent.)
\newline
\newline
\noindent
{\bf Answer:}
\newline
\[P(A) = \frac{N(E)}{S(E)} = \frac{611}{1000} = 61.1\%\]
\newline

\noindent
(3.3) How many integers from 1 through 1000 are neither multiples of 2 nor multiples of 9?
\newline
\newline
\noindent
{\bf Answer:}\\
\[1-61.1\% = 38.9\%\]
\newpage

\section*{Problem 4.}

A programmer Sam writes 500 lines of (correct) computer code in 17 days. His
boss claims that among one of the 17 days Sam must have written at least 30
lines of codes. Explain why his boss can claim that.
\newline
\newline
\noindent
{\bf Answer:}\\
The average rate of code writing will be $\frac{500}{17} = 29.4$. Assuming that you can't end the day with a partial line of code written, we can use the pigeonhole principle to say that there had to be at least one "pigeon hole" that had 30 "pigeons" fly into it. Or, there was at least one day that 30 lines were created.
\newline
\newpage

\section*{Problem 5.}

\noindent
(5.1) How many {\color{red}17}-bit strings contain exactly eight {\color{red}1}'s?
\newline
\newline
\noindent
{\bf Answer:}\\
\[{17 \choose 8} = 24310\]
\newline
\newline

\noindent
(5.2) How many {\color{red}17}-bit strings contain at least fourteen {\color{red}1}'s?
\newline
\newline
\noindent
{\bf Answer:}\\
\[{17 \choose 14} + {17 \choose 15} + {17 \choose 16} +{17 \choose 17} = 680 + 136 + 17 + 1 = 834\]
\newline
\newline

\noindent
(5.3) How many {\color{red}17}-bit strings contain at least one {\color{red}1}?
\newline
\newline
\noindent
{\bf Answer:}\\
\[2^{17} - 1 = 131071\]
\newline
\newline

\noindent
(5.4) How many {\color{red}17}-bit strings contain at most one {\color{red}1}?
\newline
\newline
\noindent
{\bf Answer:}\\
\[17 + 1 = 18\]

\end{document}
